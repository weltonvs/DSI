%%%%%%%%%%%%%%%%%%%%%%%%%%%%%%%%%%%%%%%%%%%%%%%%%%%%%%%%%%%%%%%%%%%%%%%%
% Plantilla TFG/TFM
% Universidad de A Coruña. Facultad de Informática
% Realizado por: Welton Vieira dos Santos
% Modificado: Welton Vieira dos Santos
% Contacto: welton.dossantos@udc.es
%%%%%%%%%%%%%%%%%%%%%%%%%%%%%%%%%%%%%%%%%%%%%%%%%%%%%%%%%%%%%%%%%%%%%%%%


\chapter{Análisis de Viabilidad: Modelado de la Organización.}

\section{Formulario OM-1: Contexto Organizacional, problemas y soluciones.}
Identificación de los problemas y oportunidades orientadas al conocimiento de la organización, como se muestra en la Tabla \ref{tab:OM1}.


\begin{table}[H]
\scriptsize
\begin{tabularx}{\textwidth}{|l|X|} \hline
\textbf{Modelo de Organización} & \textbf{Formulario OM-1: Problemas y Posibilidades de Mejora} \\ \hline\hline

\textsc{Problemas y Oportunidades} & Invertir o especular con activos financieros en bolsa de valores es una labor bastante compleja, ya que si una persona no tiene el conocimiento suficiente y adecuado puede ser un auténtico ``desastre''. Con ese panorama, se presenta una oportunidad para desarrollar sistemas que sean capaces de hacer esa labor con el menor error posible y con ganancias muy superior a media de operaciones hechas actualmente en el mercado financiero. Además de acercar esa práctica de inversión a particulares (conocidos como inversores minoritarios), que generalmente no tiene mucho capital para hacer inversiones mas seguras.\\ \hline
\textsc{Contexto Organizacional} & 
\begin{enumerate}
  \item La Misión, visión y objetivos de la organización son:  
  \begin{enumerate}
    \item \textbf{Misión:} Una pequeña empresa (Startup) de desarrollo de aplicaciones inteligentes relacionadas con el mercado financiero.
    \item \textbf{Visión:} Dando continuidad de un sistema inteligente desarrollado anteriormente, donde el sistema sólo lidiaba con mercados de acciones y ahora funcionará con los mercados de divisas (Forex).
    \item \textbf{Objetivos:} Seguir creciendo como empresa desarrolladora y seguir expandiendo las ventas de nuestras soluciones inteligentes para que en el futuro se pueda comercializar de forma globalizada.
  \end{enumerate}
  \item Estrategia de organización: desenvolver un producto de sofware con un coste accesible para que tenga un gran público.
  \item Escala de valores por la que se rige: obtener el máximo benefício que el mercado financiero pueda ofrecer.
\end{enumerate}    \\ \hline
\textsc{Soluciones} & La solución que se propone es desarrollar un sistema bajo coste que sea eficaz y eficiente para que un inversor minorista pueda especular en los mercados de divisa (Forex) de una forma segura y con unos beneficios razonables con respecto al riesgo del capital invertido. \\
\hline
\end{tabularx}
  %\label{tab.OM1}
  \caption{\label{tab:OM1}Contexto Organizacional - OM1}
\end{table}
	
 
%%%%%%%%%%%%%%%%%%%%%%%%%%%%%%%%%%%%%%%%%%%%%%%%%%%%%%%%%%%%%%%%%%%%%%%%%%%%%%%
\section{Formulario OM-2: descripción del área de interés de la organización.}

Descripción de los aspectos de la organización que tienen impacto y/o se ven afectados
  por las soluciones basadas en conocimiento elegidas.


\begin{table}[H]
\scriptsize
\begin{tabularx}{\textwidth}{|l|X|} \hline
\textbf{Modelo de Organización} & \textbf{Formulario OM-2: Aspectos Variables} \\ \hline\hline

\textsc{Estructura} & Se presenta un gráfico de la parte de la organización bajo análisis en términos de
departamentos, grupos, unidades\ldots\\ \hline
\textsc{Procesos} & Se realiza un diagrama de los procesos que se llevan a cabo. Posteriormente se
presentará una descomposición de estos procesos en tareas en el formulario OM-3\\ \hline
\textsc{Personal} &  Se indica qué miembros de la plantilla están implicados en los procesos, como demandantes, destinatarios o proveedores de ese conocimiento. No tienen que ser necesariamente personas físicas, sino que pueden especificarse por el rol
funcional que desarrollan en la organización (director, secretaria, etc.)\\ \hline
\textsc{Recursos} &  Describir los recursos utilizados por los procesos:
\begin{enumerate}
    \item Sistemas de información y otros recursos computacionales.
    \item Equipamiento y material.
    \item Experiencia social o interpersonal que no sea intensiva en conocimiento.
    \item Tecnología, patentes, etc.
\end{enumerate}
\\ \hline
\textsc{Conocimiento} &  Representa un recurso especial explotado en el proceso. Debido a la
importancia de este aspecto este componente dispone de un formulario aparte (OM-4)\\ \hline
\textsc{Cultura y Potencial} &  En este apartado se trata de reflejar aquellos \textit{modus operandi} que no están explícitos,
incluyendo formas de trabajar, de comunicarse y relaciones formales e informales.\\ \hline
\end{tabularx}
  \label{tab.OM2}
\end{table}
 


 






