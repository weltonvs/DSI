%%%%%%%%%%%%%%%%%%%%%%%%%%%%%%%%%%%%%%%%%%%%%%%%%%%%%%%%%%%%%%%%%%%%%%%%
% Plantilla TFG/TFM
% Universidad de A Coruña. Facultad de Informática
% Realizado por: Welton Vieira dos Santos
% Modificado: Welton Vieira dos Santos
% Contacto: welton.dossantos@udc.es
%%%%%%%%%%%%%%%%%%%%%%%%%%%%%%%%%%%%%%%%%%%%%%%%%%%%%%%%%%%%%%%%%%%%%%%%


\chapter{Análisis de Impactos y Mejoras: Modelado de las Tareas y los Agentes.}
\section{Formulario TM-1: análisis de tareas.}
Descripción detallada de tareas en el contexto del proceso de interés.

\begin{table}[H]
	\scriptsize
	\begin{tabularx}{\textwidth}{|l|X|} 
		\hline
	
		\textbf{Modelo de Tareas} & \textbf{Formulario TM-1: Análisis de Tareas} \\ 
		\hline\hline
		\textsc{Tarea} & Tarea 1. Análisis del mercado\\ 
		\hline
		\textsc{Organización}  & Departamento de Análisis.\\ 
		\hline
		\textsc{Objetivo y valor} &  Es una parte esencial de la expeculación en los mercados de divisas.\\ 
		\hline
		\textsc{Dependencia y Flujos} & \textit{1. Tareas precedentes:} Ninguna\\ &  \textit{2. Tareas que le siguen:} Tarea 3 o 4 \\
		\hline
		\textsc{Objetos manipulados} & \textit{1. Objetos de entrada de la tarea:} Información de los precios del mercado (a través de las herramientas de lectura de mercado), Información recibida por el sistema inteligente de predicción (desarrollaldo en un proyecto anterior).\\ & \textit{2. Objetos de salida de la tarea:} información respecto a la tendencia de los precios y un indicador de del precio futuro.\\  & \textit{3. Objetos internos:} conocimiento de los  expertos inversores en bolsa. \\ & \emph{Todos estos objetos incluyen elementos de información y conocimiento.}\\
		\hline
		\textsc{Tiempo y control} & \textit{1. Frecuencia y duración:} es una tarea que se da cuando se considera  oportuna (en función de la información y conocimiento del sistema).\\ & \textit{2. Control:} respecto a otras  tareas, ninguna.\\ & \textit{3. Restricciones:} Se necesita una conexión permanente con el ``Bróker''. \\
		\hline
		\textsc{Agentes} & Inversor (usuario).\\
		\hline
		\textsc{Conocimiento y Capacidad} & Experiencia en análisis de mercado de divisas (Forex). \\
		\hline
		\textsc{Recursos} & Herramientas de lectura de mercado, Sistema inteligente de predicción y  Base de Datos de precio del activo analizado. \\
		\hline
		\textsc{Calidad y eficiencia} & Se evalúan los resultados de los indicadores de predicción y de la tendencia de los precios. \\
		\hline
	\end{tabularx}
	\caption{\label{tab:TM1T1}Formulario TM-1: Analisis de tarea 1 del OM-3}
\end{table} 


\begin{table}[H]
	\scriptsize
	\begin{tabularx}{\textwidth}{|l|X|} 
		\hline	
		\textbf{Modelo de Tareas} & \textbf{Formulario TM-1: Análisis de Tareas} \\ 
		\hline\hline
		\textsc{Tarea} & Tarea 3. Operación Comprar.\\ 
		\hline
		\textsc{Organización}  & Departamento de Análisis.\\ 
		\hline
		\textsc{Objetivo y valor} &  Es una parte esencial de la expeculación en los mercados de divisas.\\ 
		\hline
		\textsc{Dependencia y Flujos} & \textit{1. Tareas precedentes:} Tarea 1 y 2\\ &  \textit{2. Tareas que le siguen:} Tarea 5. \\
		\hline
		\textsc{Objetos manipulados} & \textit{1. Objetos de entrada de la tarea:} datos introducidos por el usuario, información recibida por el sistema inteligente de predicción.\\ & \textit{2. Objetos de salida de la tarea:} Ejecución de la orden de compra al mercado.\\  & \textit{3. Objetos internos:} conocimiento de los  expertos expeculadores de mercados de divisas. \\ & \emph{Todos estos objetos incluyen elementos de información y conocimiento.}\\
		\hline
		\textsc{Tiempo y control} & \textit{1. Frecuencia y duración:} es una tarea que se da cuando se considera  oportuna (en función de la información y conocimiento del sistema).\\ & \textit{2. Control:} respecto a otras  tareas, ninguna.\\ & \textit{3. Restricciones:} Se necesita una conexión permanente con el ``Bróker''. \\
		\hline
		\textsc{Agentes} & Inversor (usuario).\\
		\hline
		\textsc{Conocimiento y Capacidad} & Experiencia en compra/venta de activos en mercado de divisas. \\
		\hline
		\textsc{Recursos} & Sistema inteligente de predicción y  base de datos. \\
		\hline
		\textsc{Calidad y eficiencia} & Se evalúan los beneficios obtenidos para determinar su calidad y eficiencia. \\
		\hline
	\end{tabularx}
	\caption{\label{tab:TM1T3}Formulario TM-1: Analisis de tarea 3 del OM-3}
\end{table}

\begin{table}[H]
	\scriptsize
	\begin{tabularx}{\textwidth}{|l|X|} 
		\hline	
		\textbf{Modelo de Tareas} & \textbf{Formulario TM-1: Análisis de Tareas} \\ 
		\hline\hline
		\textsc{Tarea} & Tarea 4. Operación Vender.\\ 
		\hline
		\textsc{Organización}  & Departamento de Análisis.\\ 
		\hline
		\textsc{Objetivo y valor} &  Es una parte esencial de la expeculación en los mercados de divisas.\\ 
		\hline
		\textsc{Dependencia y Flujos} & \textit{1. Tareas precedentes:} Tarea 1 y 2\\ &  \textit{2. Tareas que le siguen:} Tarea 5. \\
		\hline
		\textsc{Objetos manipulados} & \textit{1. Objetos de entrada de la tarea:} datos introducidos por el usuario, información recibida por el sistema inteligente de predicción.\\ & \textit{2. Objetos de salida de la tarea:} Ejecución de la orden de venta al mercado.\\  & \textit{3. Objetos internos:} conocimiento de los  expertos expeculadores de mercados de divisas. \\ & \emph{Todos estos objetos incluyen elementos de información y conocimiento.}\\
		\hline
		\textsc{Tiempo y control} & \textit{1. Frecuencia y duración:} es una tarea que se da cuando se considera  oportuna (en función de la información y conocimiento del sistema).\\ & \textit{2. Control:} respecto a otras  tareas, ninguna.\\ & \textit{3. Restricciones:} Se necesita una conexión permanente con el ``Bróker''. \\
		\hline
		\textsc{Agentes} & Inversor (usuario).\\
		\hline
		\textsc{Conocimiento y Capacidad} & Experiencia en compra/venta de activos en mercado de divisas. \\
		\hline
		\textsc{Recursos} & Sistema inteligente de predicción y  base de datos. \\
		\hline
		\textsc{Calidad y eficiencia} & Se evalúan los beneficios obtenidos para determinar su calidad y eficiencia. \\
		\hline
	\end{tabularx}
	\caption{\label{tab:TM1T4}Formulario TM-1: Analisis de tarea 4 del OM-3}
\end{table} 

\begin{table}[H]
	\scriptsize
	\begin{tabularx}{\textwidth}{|l|X|} 
		\hline	
		\textbf{Modelo de Tareas} & \textbf{Formulario TM-1: Análisis de Tareas} \\ 
		\hline\hline
		\textsc{Tarea} & Tarea 5. Gestionar las operativas abiertas.\\ 
		\hline
		\textsc{Organización}  & Departamento de Análisis.\\ 
		\hline
		\textsc{Objetivo y valor} &  Es una parte esencial de la expeculación en los mercados de divisas.\\ 
		\hline
		\textsc{Dependencia y Flujos} & \textit{1. Tareas precedentes:} Tarea 3 o 4\\ &  \textit{2. Tareas que le siguen:} Ninguna \\
		\hline
		\textsc{Objetos manipulados} & \textit{1. Objetos de entrada de la tarea:} Información de la operativas abiertas al mercado, información recibida por el sistema inteligente de predicción.\\ & \textit{2. Objetos de salida de la tarea:} Ejecución de la gestión de riesgo de la operativa abierta y en su caso cerrar la posición abierta.\\  & \textit{3. Objetos internos:} conocimiento de los  expertos expeculadores de mercados de divisas. \\ & \emph{Todos estos objetos incluyen elementos de información y conocimiento.}\\
		\hline
		\textsc{Tiempo y control} & \textit{1. Frecuencia y duración:} es una tarea que se da cuando se considera  oportuna (en función de la información y conocimiento del sistema).\\ & \textit{2. Control:} respecto a otras  tareas, ninguna.\\ & \textit{3. Restricciones:} Se necesita una conexión permanente con el ``Bróker''. \\
		\hline
		\textsc{Agentes} & Inversor (usuario).\\
		\hline
		\textsc{Conocimiento y Capacidad} & Experiencia en gestión de activos en mercado de divisas. Teorías de gestión de captial de inversión\\
		\hline
		\textsc{Recursos} & Sistema inteligente de predicción y  base de datos. \\
		\hline
		\textsc{Calidad y eficiencia} & Se evalúan los beneficios obtenidos para determinar su calidad y eficiencia. \\
		\hline
	\end{tabularx}
	\caption{\label{tab:TM1T4}Formulario TM-1: Analisis de tarea 5 del OM-3}
\end{table} 

\section{Formulario TM-2: análisis de los cuellos de botella del conocimiento.}
Especificación del conocimiento que se emplea en una tarea, sus cuellos de botella y posibles mejoras.

\begin{table}[H]
	\centering
	\resizebox{17.5cm}{!}{
	  \begin{tabular}{|l|l|l|} 
		\hline
		\textbf{Modelo de Tareas} & \multicolumn{2}{p{15.0cm}|}{\textbf{Formulario TM-2: Elementos de conocimiento}}\\ 
		\hline\hline

		\textsc{Nombre} & \multicolumn{2}{p{15.0cm}|}{Experiencia en gestionar las operativas de compra y venta de activos al mercado de divisas} \\
		\hline

		\textsc{Poseído por} & \multicolumn{2}{p{15.0cm}|}{Expertos en especulación en mercados de activos} \\
		\hline

		\textsc{Usado en} & \multicolumn{2}{p{15.0cm}|}{Tarea 5 - Gestionar las operativas abiertas} \\
		\hline

		\textsc{Dominio} & \multicolumn{2}{p{15.0cm}|}{Inversión y especulación en bolsa, ámbito económico} \\
		\hline

		\textsc{\textbf{Naturaleza del conocimiento}} & \multicolumn{1}{p{1.2cm}|}{\centering \textit{\textbf{Si/No}}} & \multicolumn{1}{p{13.0cm}|}{\textbf{¿Cuello de botella/debe ser mejorado?}}\\
		\hline

		Formal, riguroso & \multicolumn{1}{p{1.0cm}|}{Si} & \multicolumn{1}{p{13.0cm}|}{No}\\
		\hline

		Empírico, cuantitativo & \multicolumn{1}{p{1.0cm}|}{Si} & \multicolumn{1}{p{13.0cm}|}{No}\\
		\hline

		Heurístico, sentido común & \multicolumn{1}{p{1.0cm}|}{Si} & \multicolumn{1}{p{13.0cm}|}{Si, no es fácil de transferir, si es mejorable}\\
		\hline

		\multicolumn{1}{|p{6.0cm}|}{Altamente especializado, específico del dominio} & \multicolumn{1}{p{1.0cm}|}{Si} & \multicolumn{1}{p{13.0cm}|}{Si, se necesita conocimientos del dominio, es mejorable}\\
		\hline

		Basado en la experiencia & \multicolumn{1}{p{1.0cm}|}{Si} & \multicolumn{1}{p{13.0cm}|}{Si, hay una dependencia enorme del experto,es mejorable}\\
		\hline

		Basado en la acción & \multicolumn{1}{p{1.0cm}|}{No} & \multicolumn{1}{p{13.0cm}|}{No}\\
		\hline

		Incompleto & \multicolumn{1}{p{1.0cm}|}{No} & \multicolumn{1}{p{13.0cm}|}{No}\\
		\hline

		Incierto, puede ser incorrecto & \multicolumn{1}{p{1.0cm}|}{Si} & \multicolumn{1}{p{13.0cm}|}{Si, pero es impredicible, no es mejorable}\\
		\hline

		Cambia con rapidez & \multicolumn{1}{p{1.0cm}|}{No} & \multicolumn{1}{p{13.0cm}|}{No, pero es cierto que se hay que adaptar al mercado.}\\
		\hline

		Dificil de verificar & \multicolumn{1}{p{1.0cm}|}{No} & \multicolumn{1}{p{13.0cm}|}{No}\\
		\hline

		Tácito, dificil de transferir& \multicolumn{1}{p{1.0cm}|}{No} & \multicolumn{1}{p{13.0cm}|}{Si, hay que encontrar una forma de plasmarlo}\\
		\hline

		\textsc {\textbf{Forma del conocimiento}}& \multicolumn{1}{p{1.0cm}|}{} & \multicolumn{1}{p{13.0cm}|}{}\\
		\hline

		Mental & \multicolumn{1}{p{1.0cm}|}{Si} & \multicolumn{1}{p{13.0cm}|}{Si, debemos pasar a un medio que nos permita trabajar con él}\\
		\hline

		Papel & \multicolumn{1}{p{1.0cm}|}{Si} & \multicolumn{1}{p{13.0cm}|}{No}\\
		\hline

		Electrónica & \multicolumn{1}{p{1.0cm}|}{No} & \multicolumn{1}{p{13.0cm}|}{}\\
		\hline

		Habilidades & \multicolumn{1}{p{1.0cm}|}{Si} & \multicolumn{1}{p{13.0cm}|}{No}\\
		\hline

		Otros & \multicolumn{1}{p{1.0cm}|}{No} & \multicolumn{1}{p{13.0cm}|}{}\\
		\hline

		\textsc {\textbf{Disponibilidad del Conocimiento}} & \multicolumn{1}{p{1.0cm}|}{} & \multicolumn{1}{p{13.0cm}|}{}\\
		\hline
		Limitaciones de tiempo& \multicolumn{1}{p{1.0cm}|}{Si} & \multicolumn{1}{p{13.0cm}|}{Si, dependemos de los expertos}\\
		\hline

		Limitaciones de espacio& \multicolumn{1}{p{1.0cm}|}{No} & \multicolumn{1}{p{13.0cm}|}{}\\
		\hline

		Limitaciones de acceso& \multicolumn{1}{p{1.0cm}|}{Si} & \multicolumn{1}{p{13.0cm}|}{Si, dependemos de los expertos}\\
		\hline

		Limitaciones de calidad& \multicolumn{1}{p{1.0cm}|}{Si} & \multicolumn{1}{p{13.0cm}|}{Si, depende de la calidad de los expertos}\\
		\hline

		Limitaciones de forma& \multicolumn{1}{p{1.0cm}|}{No} & \multicolumn{1}{p{13.0cm}|}{}\\
		\hline

	  \end{tabular}
	}
	\caption{\label{tab:TM2}Formulario TM-2: Analisis de cuellos de botella}
  \end{table}

\section{Formulario AM-1: descripción de los agentes.}
Descripción de los agentes implicados en las tareas de interés.

\begin{table}[H]
	\centering
	\resizebox{16.0cm}{!}{
	  \begin{tabular}{|l|l|} 
		\hline
		\textbf{Modelo de Agentes} & \textbf{Formulario TM-1: Agente}\\ 
		\hline\hline
		\textsc{Nombre} & \multicolumn{1}{p{15.0cm}|}{Experto en especulación mercado activos} \\
		\hline

		\textsc{Organización} & \multicolumn{1}{p{15.0cm}|}{Departamento de Análisis}\\
		\hline

		\textsc{Implicado en} & \multicolumn{1}{p{15.0cm}|}{Todas las tareas}\\
		\hline

		\textsc{Se comunica con} & \multicolumn{1}{p{15.0cm}|}{Otros agentes (Sistema de predicción) y Herramientas de Análisis}\\
		\hline

		\textsc{Conocimiento} & \multicolumn{1}{p{15.0cm}|}{El conocimiento que tiene sobre el proceso es elevado}\\
		\hline

		\textsc{Otras competencias} & \multicolumn{1}{p{15.0cm}|}{-}\\
		\hline

		\textsc{Responsabilidades y restricciones} & \multicolumn{1}{p{15.0cm}|}{Maximizar ganancias y minimizar pérdidas}\\
		\hline

	  \end{tabular}
	}
	\caption{\label{tab:AM}Formulario AM-1: Analisis de agentes}
  \end{table}






