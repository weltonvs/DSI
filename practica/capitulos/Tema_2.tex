%%%%%%%%%%%%%%%%%%%%%%%%%%%%%%%%%%%%%%%%%%%%%%%%%%%%%%%%%%%%%%%%%%%%%%%%
% Plantilla TFG/TFM
% Universidad de A Coruña. Facultad de Informática
% Realizado por: Welton Vieira dos Santos
% Modificado: Welton Vieira dos Santos
% Contacto: welton.dossantos@udc.es
%%%%%%%%%%%%%%%%%%%%%%%%%%%%%%%%%%%%%%%%%%%%%%%%%%%%%%%%%%%%%%%%%%%%%%%%


\chapter{Análisis de Impactos y Mejoras: Modelado de las Tareas y los Agentes.}
\section{Formulario TM-1: análisis de tareas.}
Descripción detallada de tareas en el contexto del proceso de interés.

\begin{table}[H]
	\scriptsize
	\begin{tabularx}{\textwidth}{|l|X|} 
		\hline
	
		\textbf{Modelo de Tareas} & \textbf{Formulario TM-1: Análisis de Tareas} \\ 
		\hline\hline
		\textsc{Tarea} & Tarea 1. Análisis del mercado\\ 
		\hline
		\textsc{Organización}  & Departamento de Análisis.\\ 
		\hline
		\textsc{Objetivo y valor} &  Es una parte esencial de la expeculación en los mercados de divisas.\\ 
		\hline
		\textsc{Dependencia y Flujos} & \textit{1. Tareas precedentes:} Ninguna\\ &  \textit{2. Tareas que le siguen:} Tarea 3 o 4 \\
		\hline
		\textsc{Objetos manipulados} & \textit{1. Objetos de entrada de la tarea:} Información de los precios del mercado (a través de las herramientas de lectura de mercado), Información recibida por el sistema inteligente de predicción (desarrollaldo en un proyecto anterior).\\ & \textit{2. Objetos de salida de la tarea:} información respecto a la tendencia de los precios y un indicador de del precio futuro.\\  & \textit{3. Objetos internos:} conocimiento de los  expertos inversores en bolsa. \\ & \emph{Todos estos objetos incluyen elementos de información y conocimiento.}\\
		\hline
		\textsc{Tiempo y control} & \textit{1. Frecuencia y duración:} es una tarea que se da cuando se considera  oportuna (en función de la información y conocimiento del sistema).\\ & \textit{2. Control:} respecto a otras  tareas, ninguna.\\ & \textit{3. Restricciones:} Se necesita una conexión permanente con el ``Bróker''. \\
		\hline
		\textsc{Agentes} & Inversor (usuario).\\
		\hline
		\textsc{Conocimiento y Capacidad} & Experiencia en análisis de mercado de divisas (Forex). \\
		\hline
		\textsc{Recursos} & Herramientas de lectura de mercado, Sistema inteligente de predicción y  Base de Datos de precio del activo analizado. \\
		\hline
		\textsc{Calidad y eficiencia} & Se evalúan los resultados de los indicadores de predicción y de la tendencia de los precios. \\
		\hline
	\end{tabularx}
	\caption{\label{tab:TM1T1}Formulario TM-1: Analisis de tarea 1 del OM-3}
\end{table} 


\begin{table}[H]
	\scriptsize
	\begin{tabularx}{\textwidth}{|l|X|} 
		\hline	
		\textbf{Modelo de Tareas} & \textbf{Formulario TM-1: Análisis de Tareas} \\ 
		\hline\hline
		\textsc{Tarea} & Tarea 3. Operación Comprar.\\ 
		\hline
		\textsc{Organización}  & Departamento de Análisis.\\ 
		\hline
		\textsc{Objetivo y valor} &  Es una parte esencial de la expeculación en los mercados de divisas.\\ 
		\hline
		\textsc{Dependencia y Flujos} & \textit{1. Tareas precedentes:} Tarea 1 y 2\\ &  \textit{2. Tareas que le siguen:} Tarea 5. \\
		\hline
		\textsc{Objetos manipulados} & \textit{1. Objetos de entrada de la tarea:} datos introducidos por el usuario, información recibida por el sistema inteligente de predicción.\\ & \textit{2. Objetos de salida de la tarea:} Ejecución de la orden de compra al mercado.\\  & \textit{3. Objetos internos:} conocimiento de los  expertos expeculadores de mercados de divisas. \\ & \emph{Todos estos objetos incluyen elementos de información y conocimiento.}\\
		\hline
		\textsc{Tiempo y control} & \textit{1. Frecuencia y duración:} es una tarea que se da cuando se considera  oportuna (en función de la información y conocimiento del sistema).\\ & \textit{2. Control:} respecto a otras  tareas, ninguna.\\ & \textit{3. Restricciones:} Se necesita una conexión permanente con el ``Bróker''. \\
		\hline
		\textsc{Agentes} & Inversor (usuario).\\
		\hline
		\textsc{Conocimiento y Capacidad} & Experiencia en compra/venta de activos en mercado de divisas. \\
		\hline
		\textsc{Recursos} & Sistema inteligente de predicción y  base de datos. \\
		\hline
		\textsc{Calidad y eficiencia} & Se evalúan los beneficios obtenidos para determinar su calidad y eficiencia. \\
		\hline
	\end{tabularx}
	\caption{\label{tab:TM1T3}Formulario TM-1: Analisis de tarea 3 del OM-3}
\end{table}

\begin{table}[H]
	\scriptsize
	\begin{tabularx}{\textwidth}{|l|X|} 
		\hline	
		\textbf{Modelo de Tareas} & \textbf{Formulario TM-1: Análisis de Tareas} \\ 
		\hline\hline
		\textsc{Tarea} & Tarea 4. Operación Vender.\\ 
		\hline
		\textsc{Organización}  & Departamento de Análisis.\\ 
		\hline
		\textsc{Objetivo y valor} &  Es una parte esencial de la expeculación en los mercados de divisas.\\ 
		\hline
		\textsc{Dependencia y Flujos} & \textit{1. Tareas precedentes:} Tarea 1 y 2\\ &  \textit{2. Tareas que le siguen:} Tarea 5. \\
		\hline
		\textsc{Objetos manipulados} & \textit{1. Objetos de entrada de la tarea:} datos introducidos por el usuario, información recibida por el sistema inteligente de predicción.\\ & \textit{2. Objetos de salida de la tarea:} Ejecución de la orden de venta al mercado.\\  & \textit{3. Objetos internos:} conocimiento de los  expertos expeculadores de mercados de divisas. \\ & \emph{Todos estos objetos incluyen elementos de información y conocimiento.}\\
		\hline
		\textsc{Tiempo y control} & \textit{1. Frecuencia y duración:} es una tarea que se da cuando se considera  oportuna (en función de la información y conocimiento del sistema).\\ & \textit{2. Control:} respecto a otras  tareas, ninguna.\\ & \textit{3. Restricciones:} Se necesita una conexión permanente con el ``Bróker''. \\
		\hline
		\textsc{Agentes} & Inversor (usuario).\\
		\hline
		\textsc{Conocimiento y Capacidad} & Experiencia en compra/venta de activos en mercado de divisas. \\
		\hline
		\textsc{Recursos} & Sistema inteligente de predicción y  base de datos. \\
		\hline
		\textsc{Calidad y eficiencia} & Se evalúan los beneficios obtenidos para determinar su calidad y eficiencia. \\
		\hline
	\end{tabularx}
	\caption{\label{tab:TM1T4}Formulario TM-1: Analisis de tarea 4 del OM-3}
\end{table} 

\begin{table}[H]
	\scriptsize
	\begin{tabularx}{\textwidth}{|l|X|} 
		\hline	
		\textbf{Modelo de Tareas} & \textbf{Formulario TM-1: Análisis de Tareas} \\ 
		\hline\hline
		\textsc{Tarea} & Tarea 5. Gestionar las operativas abiertas.\\ 
		\hline
		\textsc{Organización}  & Departamento de Análisis.\\ 
		\hline
		\textsc{Objetivo y valor} &  Es una parte esencial de la expeculación en los mercados de divisas.\\ 
		\hline
		\textsc{Dependencia y Flujos} & \textit{1. Tareas precedentes:} Tarea 3 o 4\\ &  \textit{2. Tareas que le siguen:} Ninguna \\
		\hline
		\textsc{Objetos manipulados} & \textit{1. Objetos de entrada de la tarea:} Información de la operativas abiertas al mercado, información recibida por el sistema inteligente de predicción.\\ & \textit{2. Objetos de salida de la tarea:} Ejecución de la gestión de riesgo de la operativa abierta y en su caso cerrar la posición abierta.\\  & \textit{3. Objetos internos:} conocimiento de los  expertos expeculadores de mercados de divisas. \\ & \emph{Todos estos objetos incluyen elementos de información y conocimiento.}\\
		\hline
		\textsc{Tiempo y control} & \textit{1. Frecuencia y duración:} es una tarea que se da cuando se considera  oportuna (en función de la información y conocimiento del sistema).\\ & \textit{2. Control:} respecto a otras  tareas, ninguna.\\ & \textit{3. Restricciones:} Se necesita una conexión permanente con el ``Bróker''. \\
		\hline
		\textsc{Agentes} & Inversor (usuario).\\
		\hline
		\textsc{Conocimiento y Capacidad} & Experiencia en gestión de activos en mercado de divisas. \\
		\hline
		\textsc{Recursos} & Sistema inteligente de predicción y  base de datos. \\
		\hline
		\textsc{Calidad y eficiencia} & Se evalúan los beneficios obtenidos para determinar su calidad y eficiencia. \\
		\hline
	\end{tabularx}
	\caption{\label{tab:TM1T4}Formulario TM-1: Analisis de tarea 5 del OM-3}
\end{table} 

\section{Formulario TM-2: análisis de los cuellos de botella del conocimiento.}
Especificación del conocimiento que se emplea en una tarea, sus cuellos de botella y posibles mejoras.

\section{Formulario AM-1: descripción de los agentes.}
Descripción de los agentes implicados en las tareas de interés.

 






