%%%%%%%%%%%%%%%%%%%%%%%%%%%%%%%%%%%%%%%%%%%%%%%%%%%%%%%%%%%%%%%%%%%%%%%%
% Plantilla TFG/TFM
% Universidad de A Coruña. Facultad de Informática
% Realizado por: Welton Vieira dos Santos
% Modificado: Welton Vieira dos Santos
% Contacto: welton.dossantos@udc.es
%%%%%%%%%%%%%%%%%%%%%%%%%%%%%%%%%%%%%%%%%%%%%%%%%%%%%%%%%%%%%%%%%%%%%%%%


\chapter{Análisis de Impactos y Mejoras: Modelado de las Tareas y los Agentes.}
\section{Formulario TM-1: análisis de tareas.}
Descripción detallada de tareas en el contexto del proceso de interés.

\begin{table}[H]
	\centering
	\resizebox{18.0cm}{!}{
	  \begin{tabular}{|l|l|} 
		\hline
		\textbf{Modelo de Tareas} & \textbf{Formulario TM-1: Análisis de Tareas}\\ 
		\hline\hline
		\textsc{TAREA} & \multicolumn{1}{p{15.0cm}|}{Tarea 1. 
		Análisis de mercado} \\
		\hline
		\textsc{ORGANIZACIÓN} & \multicolumn{1}{p{15.0cm}|}{}\\
		\hline
		\textsc{OBJETIVO Y VALOR} & \multicolumn{1}{p{15.0cm}|}{}\\
		\hline
		\textsc{TIEMPO Y CONTROL} & \multicolumn{1}{p{15.0cm}|}{}\\
		\hline
		\textsc{AGENTES} & \multicolumn{1}{p{15.0cm}|}{}\\
		\hline
		\textsc{COONOCIMIENTO Y CAPACIDAD} & \multicolumn{1}{p{15.0cm}|}{}\\
		\hline
		\textsc{RECURSOS} & \multicolumn{1}{p{15.0cm}|}{}\\
		\hline
		\textsc{CALIDAD Y EFICIENCIA} & \multicolumn{1}{p{15.0cm}|}{}\\
		\hline
	  \end{tabular}
	}
	\caption{\label{tab:TM1T1}Formulario TM-1: Analisis de tarea 1 del OM-3}
  \end{table}

  \begin{table}[H]
	\centering
	\resizebox{18.0cm}{!}{
	  \begin{tabular}{|l|l|} 
		\hline
		\textbf{Modelo de Tareas} & \textbf{Formulario TM-1: Análisis de Tareas}\\ 
		\hline\hline
		\textsc{TAREA} & \multicolumn{1}{p{15.0cm}|}{Tarea 3. 
		Operación Comprar} \\
		\hline
		\textsc{ORGANIZACIÓN} & \multicolumn{1}{p{15.0cm}|}{}\\
		\hline
		\textsc{OBJETIVO Y VALOR} & \multicolumn{1}{p{15.0cm}|}{}\\
		\hline
		\textsc{TIEMPO Y CONTROL} & \multicolumn{1}{p{15.0cm}|}{}\\
		\hline
		\textsc{AGENTES} & \multicolumn{1}{p{15.0cm}|}{}\\
		\hline
		\textsc{COONOCIMIENTO Y CAPACIDAD} & \multicolumn{1}{p{15.0cm}|}{}\\
		\hline
		\textsc{RECURSOS} & \multicolumn{1}{p{15.0cm}|}{}\\
		\hline
		\textsc{CALIDAD Y EFICIENCIA} & \multicolumn{1}{p{15.0cm}|}{}\\
		\hline
	  \end{tabular}
	}
	\caption{\label{tab:TM1T3}Formulario TM-1: Analisis de tarea 3 del OM-3}
  \end{table}

  \begin{table}[H]
	\centering
	\resizebox{18.0cm}{!}{
	  \begin{tabular}{|l|l|} 
		\hline
		\textbf{Modelo de Tareas} & \textbf{Formulario TM-1: Análisis de Tareas}\\ 
		\hline\hline
		\textsc{TAREA} & \multicolumn{1}{p{15.0cm}|}{Tarea 4.Operación Vender} \\
		\hline
		\textsc{ORGANIZACIÓN} & \multicolumn{1}{p{15.0cm}|}{}\\
		\hline
		\textsc{OBJETIVO Y VALOR} & \multicolumn{1}{p{15.0cm}|}{}\\
		\hline
		\textsc{TIEMPO Y CONTROL} & \multicolumn{1}{p{15.0cm}|}{}\\
		\hline
		\textsc{AGENTES} & \multicolumn{1}{p{15.0cm}|}{}\\
		\hline
		\textsc{COONOCIMIENTO Y CAPACIDAD} & \multicolumn{1}{p{15.0cm}|}{}\\
		\hline
		\textsc{RECURSOS} & \multicolumn{1}{p{15.0cm}|}{}\\
		\hline
		\textsc{CALIDAD Y EFICIENCIA} & \multicolumn{1}{p{15.0cm}|}{}\\
		\hline
	  \end{tabular}
	}
	\caption{\label{tab:TM1T4}Formulario TM-1: Analisis de tarea 4 del OM-3}
  \end{table}

  \begin{table}[H]
	\centering
	\resizebox{18.0cm}{!}{
	  \begin{tabular}{|l|l|} 
		\hline
		\textbf{Modelo de Tareas} & \textbf{Formulario TM-1: Análisis de Tareas}\\ 
		\hline\hline
		\textsc{TAREA} & \multicolumn{1}{p{15.0cm}|}{Tarea 5. 
		Gestionar las operativas abiertas} \\
		\hline
		\textsc{ORGANIZACIÓN} & \multicolumn{1}{p{15.0cm}|}{}\\
		\hline
		\textsc{OBJETIVO Y VALOR} & \multicolumn{1}{p{15.0cm}|}{}\\
		\hline
		\textsc{TIEMPO Y CONTROL} & \multicolumn{1}{p{15.0cm}|}{}\\
		\hline
		\textsc{AGENTES} & \multicolumn{1}{p{15.0cm}|}{}\\
		\hline
		\textsc{COONOCIMIENTO Y CAPACIDAD} & \multicolumn{1}{p{15.0cm}|}{}\\
		\hline
		\textsc{RECURSOS} & \multicolumn{1}{p{15.0cm}|}{}\\
		\hline
		\textsc{CALIDAD Y EFICIENCIA} & \multicolumn{1}{p{15.0cm}|}{}\\
		\hline
	  \end{tabular}
	}
	\caption{\label{tab:TM1T5}Formulario TM-1: Analisis de tarea 5 del OM-3}
  \end{table}


\section{Formulario TM-2: análisis de los cuellos de botella del conocimiento.}
Especificación del conocimiento que se emplea en una tarea, sus cuellos de botella y posibles mejoras.

Los cuellos de botella se refieren a si la naturaleza de ese elemento de conocimiento suponen un problema a la hora de realizar la tarea correctamente. 

La segunda pregunta ``¿Debe ser mejorado'' determina si ese cuello de botella tiene o puede superarse con la implementación del sistema inteligente o, por el contrario, no es posible hacerlo. 

\section{Formulario AM-1: descripción de los agentes.}
Descripción de los agentes implicados en las tareas de interés.

\begin{table}[H]
	\centering
	\resizebox{18.0cm}{!}{
	  \begin{tabular}{|l|l|} 
		\hline
		\textbf{Modelo de Tareas} & \textbf{Formulario TM-1: Análisis de Tareas}\\ 
		\hline\hline
		\textsc{TAREA} & \multicolumn{1}{p{15.0cm}|}{Tarea 5. 
		Gestionar las operativas abiertas} \\
		\hline
		\textsc{ORGANIZACIÓN} & \multicolumn{1}{p{15.0cm}|}{}\\
		\hline
		\textsc{OBJETIVO Y VALOR} & \multicolumn{1}{p{15.0cm}|}{}\\
		\hline
		\textsc{TIEMPO Y CONTROL} & \multicolumn{1}{p{15.0cm}|}{}\\
		\hline
		\textsc{AGENTES} & \multicolumn{1}{p{15.0cm}|}{}\\
		\hline
		\textsc{COONOCIMIENTO Y CAPACIDAD} & \multicolumn{1}{p{15.0cm}|}{}\\
		\hline
		\textsc{RECURSOS} & \multicolumn{1}{p{15.0cm}|}{}\\
		\hline
		\textsc{CALIDAD Y EFICIENCIA} & \multicolumn{1}{p{15.0cm}|}{}\\
		\hline
	  \end{tabular}
	}
	\caption{\label{tab:AM}Formulario AM-1: Analisis de agentes}
  \end{table}
 






